\chapter{Предел последовательности}
\section{Определение}
Пусть имеется последовательность $a_n$. Тогда если начиная с некоторго элемента под индексом $N$ каждый следующий элемент $a_n,$ где $n>N$ будет входить в $\varepsilon$-окрестность некоторой точки $A$, то говорят, что последовательность имеет предел и он равен $A$. \newline
$\forall \varepsilon > 0\ \exists N \in \mathbb{N} : \forall n > N(n \in \mathbb{N}),\ a_n \in \mathring{U_\varepsilon}(A)$
\begin{example}
Возьмём $\displaystyle \lim_{n \to +\infty}\frac{(-1)^n}{n} = 0$
\newline
Тут $A = 0$, $a_n = \frac{(-1)^n}{n}$
\newline
Подставим значения в определение:
\begin{center}$\forall\varepsilon > 0\ \exists N \in \mathbb{N} : \forall n > N(n \in \mathbb{N}),\ \frac{(-1)^n}{n} \in \mathring{U_\varepsilon}(0)$\end{center}
$\frac{(-1)^n}{n} \in \mathring{U_\varepsilon}(0) \equiv |\frac{(-1)^n}{n} - 0| < \varepsilon$, т.к. последовательность $a_n$, принадлежащая $\varepsilon$-окрестности в точке $A = 0$ тоже самое, что расстояние между рассматриваемыми членами $a_n$ и $A = 0$ меньше $\varepsilon$.\newline
Упростим: $|\frac{(-1)^n}{n} - 0| < \varepsilon \Rightarrow |\frac{(-1)^n}{n}| < \varepsilon \Rightarrow \frac{1}{n} < \varepsilon \Rightarrow n > \frac{1}{\varepsilon} $. \newline
1)Возьмём $\varepsilon = \frac{1}{2} \Rightarrow n > 2$. Подставим в формулу наименьшее удовлетворяющее условию $n > 2$ число: $|\frac{-1^3}{3} - 0| = \frac{1}{3}$. Получается, что $\frac{1}{3} < \frac{1}{2}$ и $\forall n > 2: |\frac{(-1)^n}{n} - 0| < \varepsilon$ $\Rightarrow$ все условия из определения соблюдены. \newline
%2)Возьмём $\varepsilon = 2 \Rightarrow n > \frac{1}{2}$. Подставим в формулу наименьшее удовлетворяющие условию $n > \frac{1}{2}$ число: $|\frac{-1^1}{1} - 0| = 1$. Получается, что $1 < 2$. \textbf{Внимание!} 
\end{example}

\begin{mydef}Последовательность -- \textbf{сходящеяся}, если она имеет предел.\end{mydef}
\begin{mydef}Последовательность -- \textbf{расходящеяся}, если у нее нет предела\end{mydef}
\begin{mydef}Последовательность называется \textbf{ограниченной}, если все её члены по модулю не превосходят некоторого числа.\end{mydef}

\section{Теорема о подпоследовательности сходящей последовательности}
\begin{theorem}
Если последовательность стремится к $A$, то любая её подпоследовательность тоже стремится к $A$.\newline
$\displaystyle \lim_{x \to +\infty} a_n = A \Rightarrow \forall a_{n_k} \lim_{k \to +\infty} a_{n_k} = A$
\end{theorem}
По определению предела найдётся такой номер, что все члены с б\'oльшими номерами принадлежат $\varepsilon$-окрестности.
$\forall \varepsilon > 0\ \exists N:\ \forall n > N\ |a_n - A| < \varepsilon$\newline
Тогда $\forall k > N:\ \forall n_k > N$,\  $|a_{n_k} - A| < \varepsilon$.\newline
Значит $a_{n_k}$ стремится к $A$ по определению предела для последовательности, что и требовалось доказать.  

\section[Т. о влож. отрезках]{Теорема Коши-Кантора о вложенных отрезках}
\begin{theorem}
Для всякой системы бесконечного числа вложенных отрезков существует хотя бы одна точка, принадлежащая всем отрезкам системы.\newline
$\bigcap\limits_{n=1}^{\infty} [a_n, b_n] = c$\newline
Если длины отрезков стремятся к нулю, то такая точка \textbf{единственна}.
\end{theorem}
Обозначим за $\{ a_n\}$ множество левых концов отрезков, а за $\{ b_m \}$ -- множество правых концов.
Заметим, что $\forall n, m:\ a_n \leq b_m$. Из \textit{аксиомы непрерывности} заключаем существование точки $c$, лежащей между любыми двумя левым и правым концами:\newline
$\forall n, m\ \exists\  c: \quad a_n \leq c \leq b_m$\newline
В частности (когда $n = m$): $a_n \leq c \leq b_n$\newline 
Последнее выражение означает существование точки между концами самого маленького отрезка. Эта точка -- объединение всей системы, что и требовалось доказать.\newline\newline

Докажем единственность этой точки при стремлении длин отрезков к нулю.\newline
Пусть это не так и существуют точки $c_0,\ c_1,\ c_0 \neq c_1$. Тогда из рассуждений предыдущего доказательства следует:\newline
$(1) \quad \forall n:\ \ c_0,\ c_1 \in [a_n, b_n]$ и $|c_1 - c_0| \leq b_n - a_n$.
Т.к. длины отрезков стремятся к нулю:\newline
$(2) \quad \forall \varepsilon > 0:\ \exists N:\ \forall n > N:\ b_n - a_n < \varepsilon$ (по определению предела).\newline
Но если взять $\varepsilon = \frac{1}{2} |c_1 - c_0|$, то из $(1)$ и $(2)$ получим противоречие: $|c_1 - c_0| < \frac{1}{2} |c_1 - c_0|$.\newline
Таким образом точка $c$ единственна в случае стремления длин отрезков к нулю, что и требовалось доказать.

\section[Т. Больцано]{Теорема Больцано — Вейерштрасса}
\begin{theorem}
На любой ограниченной последовательности $x_n, n \in \mathbb{N}$ можно выделить сходящююся подпоследовательность $x_{n_k}, k \in \mathbb{N}$
\end{theorem}
Если последовательность $x_n$ ограниченна, то всё её бесконечное множество членов принадлежит некоторому промежутку, обозначим его -- $[a_0, b_0]$. Разделим этот промежуток на два равных отрезка, тогда хотя бы один из них будет содержать бесконечное число членов последовательности $x_n$, обозначим этот отрезок, как $[a_1, b_1]$. Продолжая процесс получим последовательность вложенных отрезков.
\begin{center}$[a_0, b_0] \supset [a_1, b_1] \supset [a_2, b_2] \supset \ldots$\end{center}
В которой каждый отрезок $[a_{k+1}, b_{k+1}]$ является половиной отрезка $[a_{k}, b_{k}]$ и содержит бесконечное число членов последовательности $x_n$. Т.к. размер отрезка под номером $k$ равен $S_k=\frac{|b_0-a_0|}{2^k}$, то при $k \to +\infty$, $S_k \to 0$. А по лемме о вложенных отрезках, существует единственная точка $\nu$, принадлежащая всем отрезкам. Тогда выберем подпоследовательность $x_{n_k} \in [a_k, b_k]$. Новая последовательность $x_{n_k}$ будет сходится к точке $\nu$ потому, что и $\nu$, и $x_{n_k}$ принадлежат отрезку $[a_k, b_k]$, размеры которого стремятся к $0$ при $k \to +\infty$. Т.е. $|x_{n_k}-\nu| \leq |b_k - a_k| \to 0$. Таким образом, в ограниченной последовательности $x_n$ мы выделили сходящююся подпоследовательность $x_{n_k}$.  
